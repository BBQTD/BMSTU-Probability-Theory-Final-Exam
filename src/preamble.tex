%%% Работа с русским языком
\usepackage{cmap}					% поиск в PDF
\usepackage{mathtext} 				% русские буквы в формулах
\usepackage[T2A]{fontenc}			% кодировка
\usepackage[utf8]{inputenc}			% кодировка исходного текста
\usepackage[english,russian]{babel}	% локализация и переносы
\usepackage{color} 				 	% цветные буковки
%\usepackage{indentfirst}
\frenchspacing

\renewcommand{\epsilon}{\ensuremath{\varepsilon}}
\renewcommand{\phi}{\ensuremath{\varphi}}
\renewcommand{\kappa}{\ensuremath{\varkappa}}
%\renewcommand{\le}{\ensuremath{\leqslant}}
%\renewcommand{\leq}{\ensuremath{\leqslant}}
%\renewcommand{\ge}{\ensuremath{\geqslant}}
%\renewcommand{\geq}{\ensuremath{\geqslant}}
\renewcommand{\emptyset}{\varnothing}

%%% Дополнительная работа с математикой
\usepackage{amsmath,amsfonts,amssymb,amsthm,mathtools} % AMS
\usepackage{dsfont}
\usepackage{cancel} % \cancel зачеркивание в формулах
\usepackage{mdwlist} % компактный itemize*, enumerate* and description*
\usepackage{icomma} % "Умная" запятая: $0,2$ --- число, $0, 2$ --- перечисление
\usepackage{gensymb}

%%% Перечёркивание текста \sout
\usepackage{ulem}

%% Номера формул
%\mathtoolsset{showonlyrefs=true} % Показывать номера только у тех формул, на которые есть \eqref{} в тексте.
%\usepackage{leqno} % Нумереация формул слева

%% Свои команды
\DeclareMathOperator{\sgn}{\mathop{sgn}}

%% Перенос знаков в формулах (по Львовскому)
\newcommand*{\hm}[1]{#1\nobreak\discretionary{}
{\hbox{$\mathsurround=0pt #1$}}{}}

%%% Работа с картинками
%\usepackage{floatflt}

\usepackage{graphicx}  % Для вставки рисунков
\graphicspath{{img/}{images2/}}  % папки с картинками
\setlength\fboxsep{3pt} % Отступ рамки \fbox{} от рисунка
\setlength\fboxrule{1pt} % Толщина линий рамки \fbox{}
\usepackage{wrapfig} % Обтекание рисунков текстом

%%% Работа с таблицами
\usepackage{array,tabularx,tabulary,booktabs} % Дополнительная работа с таблицами
\usepackage{longtable}  % Длинные таблицы
\usepackage{multirow} % Слияние строк в таблице

%%% Теоремы

\newtheoremstyle{customthmstyle}% ⟨name⟩ 
	{3pt}% ⟨Space above⟩ 
	{3pt}% ⟨Space below⟩
	{}% ⟨Body font⟩
	{}% ⟨Indent amount⟩1
	{\bfseries}% ⟨Theorem head font⟩
	{:}% ⟨Punctuation after theorem head⟩
	{\newline}% ⟨Space after theorem head⟩2
	{}% ⟨Theorem head spec (can be left empty, meaning ‘normal’)⟩

%\theoremstyle{customthmstyle}
%\newtheorem{definition}{Определение}[section]

%\theoremstyle{plain} % Это стиль по умолчанию, его можно не переопределять.
 
\theoremstyle{definition} % "Определение"
\newtheorem{theorem}{Теорема}[section]
\newtheorem{definition}{Определение}[section]
\newtheorem{rem}{Замечание}[definition]
\newtheorem{proposition}[theorem]{Утверждение}
\newtheorem{corollary}{Следствие}[theorem]
\newtheorem{problem}{Задача}[section]
\newtheorem*{example}{Пример}
\newtheorem{spoiler}{Рецепт}[section]
 
%\theoremstyle{remark} % "Примечание"
%\newtheorem*{nonum}{Решение}

%%% Программирование
\usepackage{etoolbox} % логические операторы

%%% Страница
\usepackage{extsizes} % Возможность сделать 14-й шрифт
\usepackage{geometry} % Простой способ задавать поля
	\geometry{top=25mm}
	\geometry{bottom=35mm}
	\geometry{left=35mm}
	\geometry{right=20mm}
 %
%\usepackage{fancyhdr} % Колонтитулы
% 	\pagestyle{fancy}
 	%\renewcommand{\headrulewidth}{0pt}  % Толщина линейки, отчеркивающей верхний колонтитул
% 	\lfoot{Нижний левый}
% 	\rfoot{Нижний правый}
% 	\rhead{Верхний правый}
% 	\chead{Верхний в центре}
% 	\lhead{Верхний левый}
%	\cfoot{Нижний в центре} % По умолчанию здесь номер страницы

\usepackage{setspace} % Интерлиньяж
%\onehalfspacing % Интерлиньяж 1.5
%\doublespacing % Интерлиньяж 2
%\singlespacing % Интерлиньяж 1

\usepackage{lastpage} % Узнать, сколько всего страниц в документе.

\usepackage{soul} % Модификаторы начертания

\usepackage{hyperref}
\usepackage[usenames,dvipsnames,svgnames,table,rgb]{xcolor}
\hypersetup{				% Гиперссылки
    unicode=true,           % русские буквы в раздела PDF
    pdftitle={Заголовок},   % Заголовок
    pdfauthor={Автор},      % Автор
    pdfsubject={Тема},      % Тема
    pdfcreator={Создатель}, % Создатель
    pdfproducer={Производитель}, % Производитель
    pdfkeywords={keyword1} {key2} {key3}, % Ключевые слова
    colorlinks=true,       	% false: ссылки в рамках; true: цветные ссылки
    linkcolor=red,          % внутренние ссылки
    citecolor=black,        % на библиографию
    filecolor=magenta,      % на файлы
    urlcolor=cyan           % на URL
}

\usepackage{csquotes} % Еще инструменты для ссылок

%\usepackage[style=authoryear,maxcitenames=2,backend=biber,sorting=nty]{biblatex}

\usepackage{multicol} % Несколько колонок

\usepackage{tikz} % Работа с графикой
\usepackage{pgfplots}
\usepackage{pgfplotstable}

% % % Оформление

% Выделение + курсив к куску текста
\newcommand{\bi}[1]{%
	\textbf{\textit{#1}}%
}

% прототип картинки
\newcommand{\imagehere}{\textcolor{blue}{*********todo: image*********}}

% проверка
\newcommand{\proverit}{\textcolor{red}{(проверить)}}


% \mathcal{text} в тексте,  проверяет включен ли математический режим и при необходимости подставляет его
\newcommand{\MathCal}[1]{%
	\ensuremath{\mathcal{#1}}%
}

% простое множество
\newcommand{\Set}[2]{%
	\ensuremath{\mathcal{#1} = \{ \mathcal{#2}_1, \dots, \mathcal{#2}_n \}}%
}

\renewcommand{\Re}{\mathds{R}}
\newcommand{\N}{\mathds{N}}

% % %

% Доказательство
\newenvironment{Proof}                    % имя окружения
{\par\noindent{$\scriptstyle\square$}}  % команды для \begin
{$\scriptstyle\blacksquare$} % команды для \end

\newcommand{\outcome}{\omega} % элементарный исход
\newcommand{\spaceoutcomes}{\Omega} % пространоство элементарных исходов
\newcommand{\sigmaalgebra}{\mathfrak{B}}
